% !TEX root = ./document.tex

\documentclass{article}

\usepackage{mystyle}
\usepackage{myvars}

%-----------------------------

\begin{document}

  \maketitle

  %-----------------------------
  %  TEXT
  %-----------------------------

  \section{Introducción}

    \paragraph{}
    La distribución ordenada es aquella que surge a partir de un conjunto de \emph{variables aleatorias independientes e igualmente distribuidas} (v.a.i.i.d. a partir de ahora) sobre las cuales se impone la restricción de orden ascendente entre ellas. De manera matemática, se dice que $X_{1}, X_{2},...,X_{i}, ..., X_{n}$ es un conjunto de v.a.i.i.d. Entonces, la distribución ordenada que surge a partir de estas se denota como $X_{(1)}, X_{(2)}, ...,X_{(i)} ,..., X_{(n)}$, es decir, se añaden paréntesis a los subíndices. Por tanto, la restricción que tiene esta distribución generada respecto de la anterior es el siguiente: $X_{(1)} <  X_{(2)} < ... < X_{(i)} < ... < X_{(n)}$.

    \paragraph{}
    Cabe destacar que estas variables llevan siguen todas ellas la misma distribución (de ahí el \emph{igualmente distribuidas}). Esto se puede denotar como $X_{i} \sim F$ donde $F$ puede ser cualquier distribución. Por tanto, todas las variables poseen la misma función de densidad y de distribución, es decir, $f(x) = f_{X_i}(x) = f_{X_j}(x)$ y $F(x) = F_{X_i}(x) = F_{X_j}(x) \ \forall i,j \in \{1,..., n\}$. La propiedad de \emph{independencia} implica por tanto, que la función de distribución conjunta cumpla la siguiente propiedad: $f(x_1, ..., x_n) = f(x_1) * ... * f(x_n)$. Dichas cualidades serán de utilidad posteriormente.

    \paragraph{}
    [TODO ]

    \paragraph{}
    El resto del trabajo se desarrolla de la siguiente manera: en las sección \ref{sec:e1} se demuestra la función de densidad conjunta para el vector $(X_{(1)}, X_{(n)})$, que después se particulariza para el caso $X_i \sim Exp(1)$. En la sección \ref{sec:e2} se estudia la distribución del rango $R = X_{(n)} - X_{(1)}$ y se particulariza para el caso exponencial de parámetro $\lambda=1$ al igual que en la sección anterior. Por último, en la sección \ref{sec:e3} se obtienen las distribuciones del rango y la mediana para el caso de que $X_i \sim U\{1,2,3\} \ \forall i  \in \{1,...,4\}$ utilizando propiedades de combinatoria.


  \section{Función de densidad conjunta de la distribución ordenada $(X_{(1)}, X_{(n)})$ para variables continuas}
  \label{sec:e2}

    \paragraph{}
    [TODO ]

    \begin{align}
      F_{(X_{(1)}, X_{(n)})} (x,y) =& \\
      =& P(X_{(1)} < x, X_{(n)} < y) \\
      =& P(X_{(n)} < y) - P(x < X_{(1)}, X_{(n)} < y) \\
      =& P(X_{(n)} < y) - P(x < X_{(1)} < X_{(n)} < y) \\
      =& P(X_{1} < y) * ... * P(X_{n} < y) - P(x < X_1 < y) * ... * P(x < X_n < y) \\
      =& F(y) * ... *F(y)  - (F(y)-F(x)) * ... * (F(y)-F(x)) \\
      =& F(y)^n - (F(y)-F(x))^n
    \end{align}

    \begin{align}
      f_{(X_{(1)}, X_{(n)})} (x,y) =& \\
      =& \frac{\partial^2}{\partial x \partial y}F_{(X_{(1)}, X_{(n)})} (x,y) \\
      =& \frac{\partial}{\partial x}( \frac{\partial}{\partial y} F(y)^n - (F(y)-F(x))^n) \\
      =& \frac{\partial}{\partial x}( n (F(y)^{n-1}*f(y) - (F(y)-F(x))^{n-1}*f(y))) \\
      =& \frac{\partial}{\partial x}( -n (F(y)-F(x))^{n-1}*f(y))) \\
      =& n(n-1)f(x)f(y)(F(y) - F(x))^{n-2}
    \end{align}

    \subsection{Particularización para $X_i \sim Exp(1)$}


      \begin{align}
        X_i \sim& Exp(1)\\
        f_{X_{i}}(x) =& e^{-x}\\
        F_{X_{i}}(x) =& 1-e^{-x} \\
      \end{align}

      \begin{align}
        f_{(X_{(1)}, X_{(n)})} (x,y) =& \\
        =& n(n-1)f(x)f(y)(F(y) - F(x))^{n-2} \\
        =& n  (n-1) e^{-x} e^{-y} ( (1 - e^{-y}) - (1 - e^{-x}) )^{n-2} \\
        =& n  (n-1) e^{-(x+y)} ( e^{-x} - e^{-y} )^{n-2}
      \end{align}

      \paragraph{}
      [TODO ]

  \section{Función de densidad del rango de la distribución ordenada $R = X_{(n)} - X_{(1)}$ para variables continuas}
  \label{sec:e2}

    \paragraph{}
    [TODO ]

    \begin{align}
      T^{-1}:
      \begin{cases}
        R = X_{(n)} - X_{(1)} \\
        S = X_{(1)}
      \end{cases} &&
      T:
      \begin{cases}
        X_{(1)} = S \\
        X_{(n)} = R + S
      \end{cases}
    \end{align}

    \begin{align}
      abs(J) =  abs(det(DT)) = abs\left(det\left( \begin{array}{cc}
\frac{\partial X_{(1)}}{ \partial R} & \frac{\partial X_{(n)}}{ \partial R}  \\
\frac{\partial X_{(1)}}{ \partial S} & \frac{\partial X_{(n)}}{ \partial S} \end{array} \right) \right) = abs\left(det\left( \begin{array}{cc}
0 & 1  \\
1 & 1 \end{array} \right) \right)  = \mid -1\mid = 1
    \end{align}


    \begin{align}
      f_{R,S}(r,s) =& \\
      =& f_{(X_{(1)}, X_{(n)})} (s,r+s) \\
      =& n(n-1)f(s)f(r+s)(F(r+s) - F(s))^{n-2}
    \end{align}

    \begin{align}
      f_{R} (r) =& \\
      =& \int_{-\infty}^{\infty} f_{R,S}(r,s) ds \\
      =& \int_{-\infty}^{\infty} n(n-1)f(s)f(r+s)(F(r+s) - F(s))^{n-2} (r,s) ds \\
      =& n(n-1)\int_{-\infty}^{\infty} f(s)f(r+s)(F(r+s) - F(s))^{n-2} (r,s) ds \\
    \end{align}
    \subsection{Particularización para $X_i \sim Exp(1)$}

      \paragraph{}
      [TODO ]

  \section{Distribuciones del rango y la mediana para la distribución ordenada generada por $X_i \sim U\{1,2,3\} \ \forall i  \in \{1,...,4\}$}
  \label{sec:e3}

    \paragraph{}
    [TODO]

  %-----------------------------
  %  Bibliographic references
  %-----------------------------

  \nocite{prob2017}


  \bibliographystyle{acm}
  \bibliography{bib}

\end{document}
