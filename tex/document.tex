% !TEX root = ./document.tex

\documentclass{article}

\usepackage{mystyle}
\usepackage{myvars}

%-----------------------------
\allowdisplaybreaks

\begin{document}

  \maketitle

  %-----------------------------
  %  TEXT
  %-----------------------------

  \section{Introducción}

    \paragraph{}
    La distribución ordenada es aquella que surge a partir de un conjunto de \emph{variables aleatorias independientes e igualmente distribuidas} (v.a.i.i.d. a partir de ahora) sobre las cuales se impone la restricción de orden ascendente entre ellas. De manera matemática, se dice que $X_{1}, X_{2},...,X_{i}, ..., X_{n}$ es un conjunto de v.a.i.i.d. Entonces, la distribución ordenada que surge a partir de estas se denota como $X_{(1)}, X_{(2)}, ...,X_{(i)} ,..., X_{(n)}$, es decir, se añaden paréntesis a los subíndices. Por tanto, la restricción que tiene esta distribución generada respecto de la anterior es el siguiente: $X_{(1)} <  X_{(2)} < ... < X_{(i)} < ... < X_{(n)}$.

    \paragraph{}
    Cabe destacar que estas variables llevan siguen todas ellas la misma distribución (de ahí el \emph{igualmente distribuidas}). Esto se puede denotar como $X_{i} \sim F$ donde $F$ puede ser cualquier distribución. Por tanto, todas las variables poseen la misma función de densidad y de distribución, es decir, $f(x) = f_{X_i}(x) = f_{X_j}(x)$ y $F(x) = F_{X_i}(x) = F_{X_j}(x), \quad \forall i,j \in \{1,..., n\}$. La propiedad de \emph{independencia} implica por tanto, que la función de distribución conjunta cumpla la siguiente propiedad: $f(x_1, ..., x_n) = f(x_1) * ... * f(x_n)$. Dichas cualidades serán de utilidad posteriormente.

    \paragraph{}
    [TODO ]

    \paragraph{}
    El resto del trabajo se desarrolla de la siguiente manera: en las sección \ref{sec:e1} se demuestra la función de densidad conjunta para el vector $(X_{(1)}, X_{(n)})$, que después se particulariza para el caso $X_i \sim Exp(1)$. En la sección \ref{sec:e2} se estudia la distribución del rango $R = X_{(n)} - X_{(1)}$ y se particulariza para el caso exponencial de parámetro $\lambda=1$ al igual que en la sección anterior. Por último, en la sección \ref{sec:e3} se obtienen las distribuciones del rango y la mediana para el caso de que $X_i \sim U\{1,2,3\} \ \forall i  \in \{1,...,4\}$ utilizando propiedades de combinatoria.


  \section{Función de densidad conjunta de la distribución ordenada $(X_{(1)}, X_{(n)})$ para variables continuas}
  \label{sec:e2}

    \paragraph{}
    En esta sección se va a desarrollar la expresión de la función de densidad de la distribución conjunta para el valor mínimo y máximo de una distribución ordenada arbitratia donde el mínimo viene dado por $X_{(1)}$ y el máximo por $X_{(n)}$. Por tanto, se estudiará la distribución generada por el vector aleatorio $(X_{(1)}, X_{(n)})$.

    \paragraph{}
    Para estudiar obtener la función de densidad conjunta en este caso, primeramente se va a desarrollar en la ecuación \ref{eq:joint_ordered_distribution} la función de distribución conjunta para llegar hasta una expresión dada a partir de la ley de las variables que componen el conjunto de \emph{v.a.i.i.d.}, que se asume como conocida. En este caso, la idea intuitiva está en el tercer paso, en el que aplica la \emph{ley de probabilidades totales} para conseguir la desigualdad $x < X_{(1)} < X_{(n)} < y$ que es fácil entender que es equivalente a $x < X_i < y,\quad \forall i \in \{1, ..., n\}$

    \begin{align}
    \label{eq:joint_ordered_distribution}
      \begin{split}
        F_{(X_{(1)}, X_{(n)})} (x,y) =& \\
        =& P(X_{(1)} < x, X_{(n)} < y) \\
        =& P(X_{(n)} < y) - P(x < X_{(1)}, X_{(n)} < y) \\
        =& P(X_{(n)} < y) - P(x < X_{(1)} < X_{(n)} < y) \\
        =& P(X_{1} < y) * ... * P(X_{n} < y) - P(x < X_1 < y) * ... * P(x < X_n < y) \\
        =& F(y) * ... *F(y)  - (F(y)-F(x)) * ... * (F(y)-F(x)) \\
        =& F(y)^n - (F(y)-F(x))^n
      \end{split}
    \end{align}

    \paragraph{}
    El siguiente paso para llegar a la función de densidad conjunta consiste en el desarrollo de la aplicación de derivación múltiple para llegar a la función de densidad a partir de la de distribución. Esto es posible puesto que se cumple que $f(x)= \int_{-\infty}^{\infty} F(x) dx$. Por tanto, en este caso, la derivación se basa únicamente en la aplicación de operaciones algebraicas a partir de dicha propiedad. Esto se muestra en la ecuación \ref{eq:joint_ordered_density}.

    \begin{align}
      \label{eq:joint_ordered_density}
      \begin{split}
        f_{(X_{(1)}, X_{(n)})} (x,y) =& \\
        =& \frac{\partial^2}{\partial x \partial y}F_{(X_{(1)}, X_{(n)})} (x,y) \\
        =& \frac{\partial}{\partial x}( \frac{\partial}{\partial y} F(y)^n - (F(y)-F(x))^n) \\
        =& \frac{\partial}{\partial x}( n (F(y)^{n-1}*f(y) - (F(y)-F(x))^{n-1}*f(y))) \\
        =& \frac{\partial}{\partial x}( -n (F(y)-F(x))^{n-1}*f(y))) \\
        =& n(n-1)f(x)f(y)(F(y) - F(x))^{n-2}
      \end{split}
    \end{align}

    \subsection{Particularización para $X_i \sim Exp(1)$}

      \paragraph{}
      Lo siguiente que se va a llevar a cabo es la particularización de la función de densidad conjunta de la distribución ordenada de \emph{v.a.i.i.d} para el caso de que estas sigan una distribución exponencial de parámetro $\lambda = 1$.

      \begin{align*}
        X_i \sim& Exp(1)\\
        f_{X_{i}}(x) =& e^{-x}\\
        F_{X_{i}}(x) =& 1-e^{-x} \\
      \end{align*}

      \paragraph{}
      Este desarrollo se lleva a cabo en la ecuación \eqref{eq:joint_ordered_density_exp} substituyendo las funciones de distribución y densidad de manera apropiada y aplicando operaciones algebraicas.

      \begin{align}
        \label{eq:joint_ordered_density_exp}
        \begin{split}
          f_{(X_{(1)}, X_{(n)})} (x,y) =& \\
          =& n(n-1)f(x)f(y)(F(y) - F(x))^{n-2} \\
          =& n  (n-1) e^{-x} e^{-y} ( (1 - e^{-y}) - (1 - e^{-x}) )^{n-2} \\
          =& n  (n-1) e^{-(x+y)} ( e^{-x} - e^{-y} )^{n-2}
        \end{split}
      \end{align}

  \section{Función de densidad del rango de la distribución ordenada $R = X_{(n)} - X_{(1)}$ para variables continuas}
  \label{sec:e2}

    \paragraph{}
    En esta sección se va a realizar el desarrollo necesario para la obtención de la función de densidad del rango generada a partir de la distribución ordenada de un conjunto de \emph{v.a.i.i.d.} definida como $R = X_{(n)} - X_{(1)}$. Esto se puede entender de manera sencilla puesto que $X_{(n)}$ representa el máximo y $X_{(1)}$ el mínimo. Para esto, se puede aplicar la propiedad de transformación de vectores de variables mediante la utilización del teorema de \emph{cambio de variable}. Para ello, es necesario conocer la función de distribución conjunta $f_{(X_{(1)}, X_{(n)})}(x,y)$, que se obtuvo en la sección anterior. La aplicación $T$ del cambio de variable se muestra en la ecuación \eqref{eq:variable_transformation} y el \emph{Jacobiano} de la misma se desarrolla en la ecuación \eqref{eq:variable_transformation_jacobian}.

    \begin{align}
    \label{eq:variable_transformation}
      T^{-1}:
      \begin{cases}
        R = X_{(n)} - X_{(1)} \\
        S = X_{(1)}
      \end{cases} &&
      T:
      \begin{cases}
        X_{(1)} = S \\
        X_{(n)} = R + S
      \end{cases}
    \end{align}

    \begin{align}
    \label{eq:variable_transformation_jacobian}
      abs(J) =
      abs(det(DT)) =
      abs\left(det\left( \begin{array}{cc}
        \frac{\partial X_{(1)}}{ \partial R} & \frac{\partial X_{(n)}}{ \partial R}  \\
        \frac{\partial X_{(1)}}{ \partial S} & \frac{\partial X_{(n)}}{ \partial S}
      \end{array} \right) \right) =
      abs\left(det\left( \begin{array}{cc}
        0 & 1  \\
        1 & 1
      \end{array} \right) \right)  =
      \mid -1\mid =
      1
    \end{align}

    \paragraph{}
    La función de densidad conjunta de la transformación $(R,S)$ se muestra en la ecuación \eqref{eq:joint_distribution_transformation}, que consiste en la aplicación del teorema de \emph{cambio de variable}, en este caso aplicado sobre la función de densidad de un vector aleatorio.

    \begin{align}
    \label{eq:joint_distribution_transformation}
      \begin{split}
        f_{R,S}(r,s) =& \\
        =& f_{(X_{(1)}, X_{(n)})} (s,r+s) \\
        =& n(n-1)f(s)f(r+s)(F(r+s) - F(s))^{n-2}
      \end{split}
    \end{align}

    \paragraph{}
    Lo siguiente es obtener la distribución marginal de la variable $R$, que representa el rango del conjunto de \emph{v.a.i.i.d.} para lo cual se ha aplicado la ley de probabilidades totales como $f(x) = \int_{-\infty}^{\infty} f(x,y) dy$. El desarrollo de la función de densidad del rango se ha llevado a cabo en la ecuación \eqref{eq:range_distribution_transformation}.

    \begin{align}
    \label{eq:range_distribution_transformation}
      \begin{split}
        f_{R} (r) =& \\
        =& \int_{-\infty}^{\infty} f_{R,S}(r,s) ds \\
        =& \int_{-\infty}^{\infty} n(n-1)f(s)f(r+s)(F(r+s) - F(s))^{n-2} ds \\
        =& n(n-1)\int_{-\infty}^{\infty} f(s)f(r+s)(F(r+s) - F(s))^{n-2} ds \\
      \end{split}
    \end{align}


    \subsection{Particularización para $X_i \sim Exp(1)$}

      \paragraph{}
      Al igual que para el caso de la función de densidad de densidad conjunta para la distribución ordenada, que se particularizó para la distribución exponencial con parámetro $\lambda = 1$, en este caso se ha hecho lo mismo para la distribución del rango.

      \begin{align*}
        X_i \sim& Exp(1)\\
        f_{X_{i}}(x) =& e^{-x}\\
        F_{X_{i}}(x) =& 1-e^{-x} \\
      \end{align*}

      \paragraph{}
      Para ello se ha utilizado la expresión de la función de densidad del rango obtenida en la ecuación \eqref{eq:range_distribution_transformation}, utilizando las funciones de distribución y densidad de la distribución exponencial de parámetro $\lambda = 1$, que se muestran arriba. Esto se muestra en la ecuación \eqref{eq:range_distribution_transformation_exp}.

      \begin{align}
      \label{eq:range_distribution_transformation_exp}
        \begin{split}
          f_{R} (r) =&\\
          =& \int_{-\infty}^{\infty} f_{R,S}(r,s) ds \\
          =& n(n-1)\int_{0}^{\infty} f(s)f(r+s)(F(r+s) - F(s))^{n-2} ds \\
          =& n(n-1)\int_{0}^{\infty} e^{-s}e^{-(r+s)}((1-e^{-(r+s)}) - (1-e^{-s}))^{n-2} ds \\
          =& n(n-1)e^{-r}\int_{0}^{\infty} e^{-2s}( e^{-s} - e^{-(r+s)})^{n-2} ds \\
          =& (n-1)e^{-r}e^{-r(-2 + n)} (e^r -1)^{-2 + n} \\
          =& (n-1)e^{-2r(-2 + n) r} (e^r -1)^{-2 + n}
        \end{split}
      \end{align}

  \section{Distribuciones del rango y la mediana para la distribución ordenada generada por $X_i \sim U\{1,2,3\} \quad \forall i  \in \{1,...,4\}$}
  \label{sec:e3}

    \paragraph{}
    En esta sección se van a obtener las leyes del rango y la mediana para el caso concreto de que el conjunto de \emph{v.a.i.i.d.} siga una distribución $X_i \sim U\{1,2,3\} \quad \forall i  \in \{1,...,4\}$. Nótese que en este caso se está trabajando con una distribución discreta, para lo cual es más apropiado apoyarse en una técnica diferente de las descritas en secciones anteriores. En este caso se utilizarán propiedades de combinatoria para facilitar la obtención de dichas leyes.

    \paragraph{}
    La formula que se utilizará para aplicar las propiedades de combinatoria para la obtención de las leyes de probabilidad será la de \emph{permutaciones con repetición} de $n$ elementos con $k$ grupos de $n_i$ elementos para cada grupo, que se muestra en la ecuación \eqref{eq:repeat_permutations}.

    \begin{equation}
    \label{eq:repeat_permutations}
      Pr(n, n_1, ..., n_i,...,n_k) = \frac{n!}{n_1! *... *n_i! * ... * n_k!}
    \end{equation}

    \paragraph{}
    En este caso se trabaja con el conjunto de \emph{v.a.i.i.d.} $X_i \sim U\{1,2,3\} \quad \forall i  \in \{1,...,4\}$, cuyo número de combinaciones distintas se basa en \emph{variaciones con repetición} donde $n = 3$ y $k=4$ por lo que hay un total de $n^k = 3^4 = 81$ posibles combinaciones.

    \subsection{Rango $R=X_{(4)}-X_{(1)}$ de $X_i \sim U\{1,2,3\} \quad \forall i  \in \{1,...,4\}$}

      \paragraph{}
      Para el caso del rango, lo primero que necesitaremos es rellenar la tabla \ref{table:joint_1_4}, en la cual se muestra la ley de probabilidad conjunta de las variables $X_{(1)},X_{(4)}$.

      \begin{table}[H]
        \centering
        \begin{tabular}{c | c  c  c | c}
          $X_{(1)},X_{(4)}$ & 1               & 2               & 3               &                 \\ \hline
          1                 & $\frac{1}{81}$  & $\frac{14}{81}$ & $\frac{50}{81}$ & $\frac{65}{81}$ \\
          2                 & 0               & $\frac{1}{81}$  & $\frac{14}{81}$ & $\frac{15}{81}$ \\
          3                 & 0               & 0               & $\frac{1}{81}$  & $\frac{1}{81}$  \\ \hline
                            &  $\frac{1}{81}$ & $\frac{15}{81}$ &                 & 1
        \end{tabular}
        \caption{Ley de la distribución conjunta de $(X_{(1)},X_{(4)})$}
        \label{table:joint_1_4}
      \end{table}

      \paragraph{}
      A continuación se justifica la obtención de dicha tabla:

      \begin{align*}
        X_{(1)} = 1,X_{(4)} = 1 & \Rightarrow \frac{1}{81}\\
        &1111 \rightarrow 1 \\
        X_{(1)} = 1,X_{(4)} = 2 & \Rightarrow \frac{14}{81}\\
        &1112 \rightarrow \frac{4!}{3!1!} = 4 \\
        &1122 \rightarrow \frac{4!}{2!2!} = 6  \\
        &1222 \rightarrow \frac{4!}{3!1!} = 4  \\
        X_{(1)} = 1,X_{(4)} = 3 & \Rightarrow \frac{50}{81}\\
        &1113 \rightarrow \frac{4!}{3!1!} = 4 \\
        &1133 \rightarrow \frac{4!}{2!2!} = 6  \\
        &1333 \rightarrow \frac{4!}{3!1!} = 4  \\
        &1123 \rightarrow \frac{4!}{2!} = 12 \\
        &1223 \rightarrow \frac{4!}{2!} = 12  \\
        &1233 \rightarrow \frac{4!}{2!} = 12  \\
        X_{(1)} = 2,X_{(4)} = 2 & \Rightarrow \frac{1}{81}\\
        &2222 \rightarrow 1 \\
        X_{(1)} = 2,X_{(4)} = 3 & \Rightarrow \frac{14}{81}\\
        &2223 \rightarrow \frac{4!}{3!1!} = 4 \\
        &2233 \rightarrow \frac{4!}{2!2!} = 6  \\
        &2333 \rightarrow \frac{4!}{3!1!} = 4  \\
        X_{(1)} = 3,X_{(4)} = 3 & \Rightarrow \frac{1}{81}\\
        &3333 \rightarrow 1 \\
      \end{align*}

      \paragraph{}
      Lo siguiente es contabilizar calcular la ley de probabilidad para el rango, la cual se ilustra en la tabla \ref{table:range}, que se ha obtenido mediante el conteo de casos que cumplen la propiedad de la transformación $R=X_{(4)} - X_{(1)}$.


      \begin{table}[H]
        \centering
        \begin{tabular}{ c | c c c |}
          $r_i$       & 0             & 1             & 2 \\ \hline
          $P(R = r_i)$  & $\frac{3}{81}$  & $\frac{28}{81}$ & $\frac{50}{81}$
        \end{tabular}
        \caption{Ley del rango $R=X_{(4)} - X_{(1)}$}
        \label{table:range}
      \end{table}

    \subsection{Mediana $Me = \frac{X_{(2)} + X_{(3)}}{2}$ de $X_i \sim U\{1,2,3\} \quad \forall i  \in \{1,...,4\}$}

      \paragraph{}
      Para el caso de la mediana, lo primero que necesitaremos es rellenar la tabla \ref{table:joint_2_3}, en la cual se muestra la ley de probabilidad conjunta de las variables $X_{(2)},X_{(3)}$.

      \begin{table}[H]
        \centering
        \begin{tabular}{c | c  c  c | c}
          $X_{(2)},X_{(3)}$ & 1               & 2               & 3               &                 \\ \hline
          1                 & $\frac{9}{81}$  & $\frac{18}{81}$ & $\frac{6}{81}$  &$\frac{39}{81}$  \\
          2                 & 0               & $\frac{21}{81}$ & $\frac{18}{81}$ & $\frac{27}{81}$ \\
          3                 & 0               & 0               & $\frac{9}{81}$  & $\frac{9}{81}$  \\ \hline
                            & $\frac{9}{81}$  & $\frac{27}{81}$ & $\frac{39}{81}$ & 1
        \end{tabular}
        \caption{Ley de la distribución conjunta de $X_{(2)},X_{(3)}$}
        \label{table:joint_2_3}
      \end{table}

      \paragraph{}
      A continuación se justifica la obtención de dicha tabla:

      \begin{align*}
        X_{(2)} = 1,X_{(3)} = 1 & \Rightarrow \frac{9}{81}\\
        &1111 \rightarrow 1 \\
        &1112 \rightarrow \frac{4!}{3!1!} = 4 \\
        &1113 \rightarrow \frac{4!}{3!1!} = 4 \\
        X_{(2)} = 1,X_{(3)} = 2 & \Rightarrow \frac{18}{81}\\
        &1123 \rightarrow \frac{4!}{2!} = 12 \\
        &1122 \rightarrow \frac{4!}{2!2!} = 6  \\
        X_{(2)} = 1,X_{(3)} = 3 & \Rightarrow \frac{6}{81}\\
        &1133 \rightarrow \frac{4!}{2!2!} = 6  \\
        X_{(2)} = 2,X_{(3)} = 2 & \Rightarrow \frac{21}{81}\\
        &2222 \rightarrow 1 \\
        &2223 \rightarrow \frac{4!}{3!1!} = 4 \\
        &1222 \rightarrow \frac{4!}{3!1!} = 4 \\
        &1223 \rightarrow \frac{4!}{2!} = 12 \\
        X_{(2)} = 2,X_{(3)} = 3 & \Rightarrow \frac{18}{81}\\
        &1233 \rightarrow \frac{4!}{2!} = 12 \\
        &2233 \rightarrow \frac{4!}{2!2!} = 6  \\
        X_{(2)} = 3,X_{(3)} = 3 & \Rightarrow \frac{9}{81}\\
        &3333 \rightarrow 1 \\
        &1333 \rightarrow \frac{4!}{3!1!} = 4 \\
        &2333 \rightarrow \frac{4!}{3!1!} = 4 \\
      \end{align*}

      \paragraph{}
      Lo siguiente es contabilizar calcular la ley de probabilidad para la mediana, la cual se ilustra en la tabla \ref{table:median}, que se ha obtenido mediante el conteo de casos que cumplen la propiedad de la transformación $Me = \frac{X_{(2)}+X_{(3))}}{2}$.

      \begin{table}[H]
        \centering
        \begin{tabular}{ c | c c c c c|}
          $me_i$       & 1               & $\frac{3}{2}$   & 2               & $\frac{5}{2}$   & 3               \\ \hline
          $P(Me = me_i)$  & $\frac{9}{81}$  & $\frac{18}{81}$ & $\frac{27}{81}$ & $\frac{18}{81}$ & $\frac{9}{81}$
        \end{tabular}
        \caption{Ley de la Mediana $Me = \frac{X_{(2)}+X_{(3))}}{2}$ }
        \label{table:median}
      \end{table}
  %-----------------------------
  %  Bibliographic references
  %-----------------------------

  \nocite{prob2017}


  \bibliographystyle{acm}
  \bibliography{bib}

\end{document}
